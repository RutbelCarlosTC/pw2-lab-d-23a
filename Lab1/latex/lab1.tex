%%%package list
\documentclass{article}
\usepackage[top=3cm, bottom=3cm, outer=3cm, inner=3cm]{geometry}
\usepackage{multicol}
\usepackage{graphicx}
\usepackage{url}

%\usepackage{cite}
\usepackage{hyperref}
\usepackage{array}
%\usepackage{multicol}
\newcolumntype{x}[1]{>{\centering\arraybackslash\hspace{0pt}}p{#1}}
\usepackage{natbib}
\usepackage{pdfpages}
\usepackage{multirow}
\usepackage[normalem]{ulem}
\useunder{\uline}{\ul}{}
\usepackage{svg}
\usepackage{xcolor}
\usepackage{listings}
\lstdefinestyle{ascii-tree}{
    literate={├}{|}1 {─}{--}1 {└}{+}1 
  }
\lstset{basicstyle=\ttfamily,
  showstringspaces=false,
  commentstyle=\color{red},
  keywordstyle=\color{blue}
}

%\usepackage{booktabs}
\usepackage{caption}
\usepackage{subcaption}
\usepackage{float}
\usepackage{array}

\newcolumntype{M}[1]{>{\centering\arraybackslash}m{#1}}
\newcolumntype{N}{@{}m{0pt}@{}}

%%%%%%%%%%%%%%%%%%%%%%%%%%%%%%%%%%%%%%%%%%%%%%%%%%%%%%%%%%%%%%%%%%%%%%%%%%%%
%%%%%%%%%%%%%%%%%%%%%%%%%%%%%%%%%%%%%%%%%%%%%%%%%%%%%%%%%%%%%%%%%%%%%%%%%%%%
\newcommand{\itemEmail}{rttitoca@unsa.edu.pe}
\newcommand{\itemStudent}{Rutbel Carlos Ttito Campos}
\newcommand{\itemCourse}{Programación Web 2}
\newcommand{\itemCourseCode}{20231001}
\newcommand{\itemSemester}{III}
\newcommand{\itemUniversity}{Universidad Nacional de San Agustín de Arequipa}
\newcommand{\itemFaculty}{Facultad de Ingeniería de Producción y Servicios}
\newcommand{\itemDepartment}{Departamento Académico de Ingeniería de Sistemas e Informática}
\newcommand{\itemSchool}{Escuela Profesional de Ingeniería de Sistemas}
\newcommand{\itemAcademic}{2023 - A}
\newcommand{\itemInput}{Del 10 Abril 2023}
\newcommand{\itemOutput}{Al 17 Abril 2023}
\newcommand{\itemPracticeNumber}{01}
\newcommand{\itemTheme}{Git y Github}
%%%%%%%%%%%%%%%%%%%%%%%%%%%%%%%%%%%%%%%%%%%%%%%%%%%%%%%%%%%%%%%%%%%%%%%%%%%%
%%%%%%%%%%%%%%%%%%%%%%%%%%%%%%%%%%%%%%%%%%%%%%%%%%%%%%%%%%%%%%%%%%%%%%%%%%%%

\usepackage[english,spanish]{babel}
\usepackage[utf8]{inputenc}
\AtBeginDocument{\selectlanguage{spanish}}
\renewcommand{\figurename}{Figura}
\renewcommand{\refname}{Referencias}
\renewcommand{\tablename}{Tabla} %esto no funciona cuando se usa babel
\AtBeginDocument{%
	\renewcommand\tablename{Tabla}
}

\usepackage{fancyhdr}
\pagestyle{fancy}
\fancyhf{}
\setlength{\headheight}{30pt}
\renewcommand{\headrulewidth}{1pt}
\renewcommand{\footrulewidth}{1pt}
\fancyhead[L]{\raisebox{-0.2\height}{\includegraphics[width=3cm]{img/logo_episunsa.png}}}
\fancyhead[C]{\fontsize{7}{7}\selectfont	\itemUniversity \\ \itemFaculty \\ \itemDepartment \\ \itemSchool \\ \textbf{\itemCourse}}
\fancyhead[R]{\raisebox{-0.2\height}{\includegraphics[width=1.2cm]{img/logo_abet.png}}}
\fancyfoot[L]{Est. \itemStudent}
\fancyfoot[C]{\itemCourse}
\fancyfoot[R]{Página \thepage}


% para el codigo fuente
\usepackage{listings}
\usepackage{color, colortbl}
\definecolor{dkgreen}{rgb}{0,0.6,0}
\definecolor{gray}{rgb}{0.5,0.5,0.5}
\definecolor{mauve}{rgb}{0.58,0,0.82}
\definecolor{codebackground}{rgb}{0.95, 0.95, 0.92}
\definecolor{tablebackground}{rgb}{0.8, 0, 0}

\lstset{frame=tb,
    language=bash,
    aboveskip=3mm,
    belowskip=3mm,
    showstringspaces=false,
    columns=flexible,
    basicstyle={\small\ttfamily},
    numbers=none,
    numberstyle=\tiny\color{gray},
    keywordstyle=\color{blue},
    commentstyle=\color{dkgreen},
    stringstyle=\color{mauve},
    breaklines=true,
    breakatwhitespace=true,
    tabsize=3,
    backgroundcolor= \color{codebackground},
}



\usepackage{graphicx} % Required for inserting images

\begin{document}

\vspace*{10px}

\begin{center}
	\fontsize{17}{17} \textbf{ Informe de Laboratorio \itemPracticeNumber}
\end{center}
\centerline{\textbf{\Large Tema: \itemTheme}}
%\vspace*{0.5cm}	

\begin{flushright}
	\begin{tabular}{|M{2.5cm}|N|}
		\hline
		\rowcolor{tablebackground}
		\color{white} \textbf{Nota} \\
		\hline
		\\[30pt]
		\hline
	\end{tabular}
\end{flushright}

\begin{table}[H]
	\begin{tabular}{|x{4.7cm}|x{4.8cm}|x{4.8cm}|}
		\hline
		\rowcolor{tablebackground}
		\color{white} \textbf{Estudiante} & \color{white}\textbf{Escuela} & \color{white}\textbf{Asignatura}                                        \\
		\hline
		{\itemStudent \par \itemEmail}    & \itemSchool                   & {\itemCourse \par Semestre: \itemSemester \par Código: \itemCourseCode} \\
		\hline
	\end{tabular}
\end{table}

\begin{table}[H]
	\begin{tabular}{|x{4.7cm}|x{4.8cm}|x{4.8cm}|}
		\hline
		\rowcolor{tablebackground}
		\color{white}\textbf{Laboratorio} & \color{white}\textbf{Tema} & \color{white}\textbf{Duración} \\
		\hline
		\itemPracticeNumber               & \itemTheme                 & 04 horas                       \\
		\hline
	\end{tabular}
\end{table}

\begin{table}[H]
	\begin{tabular}{|x{4.7cm}|x{4.8cm}|x{4.8cm}|}
		\hline
		\rowcolor{tablebackground}
		\color{white}\textbf{Semestre académico} & \color{white}\textbf{Fecha de inicio} & \color{white}\textbf{Fecha de entrega} \\
		\hline
		\itemAcademic                            & \itemInput                            & \itemOutput                            \\
		\hline
	\end{tabular}
\end{table}

\section{Equipos, materiales y temas utilizados}
\begin{itemize}
	\item Sistema Operativo Ubuntu 22.04.2 LTS
	\item Git 2.39.2.
	\item Cuenta en GitHub con el correo institucional.
	\item HTML Y CSS
\end{itemize}

\section{Tarea}
\begin{itemize}		
	\item Crear repositorio GitHub.
	\item Instalar Git
	\item Crear un directorio pw2-lab-d-23a en cualquier sitio.
	\item Dentro de este directorio inicializar un repositorio.
	\item Empezar a crear una página de acuerdo al Tutorial de la W3C.
	\item Agregar contenido personalizado por ud.
	\item Agregar fotos.
\end{itemize}

\section{Solucion de Ejercicios}

\subsection{Creando repositorio}
\begin{figure}[H]
    \centering
    \includegraphics[scale=0.5]{img/repo.png}
    \caption{eRepositorio pw2-2023-a Lab 1}
\end{figure}

\subsection{HTML principal}
\lstinputlisting[language=HTML, caption={index.html},numbers=left,]{src/index.html}

\subsection{Seccion Amigos}
\lstinputlisting[language=HTML, caption={amigos.html},numbers=left,]{src/amigos.html}

\subsection{Seccion Uiversidad}
\lstinputlisting[language=HTML, caption={universidad.html},numbers=left,]{src/universidad.html}

\subsection{Seccion Hobies}
\lstinputlisting[language=HTML, caption={hobbies.html},numbers=left,]{src/hobbies.html}

\subsection{Capturas}
\begin{figure}[H]
    \centering
    \includegraphics[scale=0.4]{img/exe/principal.png}
    \caption{Pagina principal}
\end{figure}
\begin{figure}[H]
    \centering
    \includegraphics[scale=0.4]{img/exe/Seccion Amigos.png}
    \caption{Seccion Amigos}
\end{figure}
\begin{figure}[H]
    \centering
    \includegraphics[scale=0.4]{img/exe/Seccion universidad.png}
    \caption{Seccion universidad}
\end{figure}
\begin{figure}[H]
    \centering
    \includegraphics[scale=0.4]{img/exe/Seccion Hobbies.png}
    \caption{Seccion Hobbies}
\end{figure}


\section{URL de Repositorio Github}
\begin{itemize}
	\item URL del Repositorio GitHub para clonar o recuperar:
	\item \url{https://github.com/RutbelCarlosTC/pw2-lab-d-23a.git}
	\item URL para el laboratorio 04 en el Repositorio GitHub.
	\item \url{https://github.com/RutbelCarlosTC/pw2-lab-d-23a/tree/main/Lab1}
\end{itemize}

\section{Actividades con el repositorio GitHub}

\subsection{Commits}
\begin{itemize}
	\item Agregando archivos iniciales:
	\begin{figure}[H]
	    \centering
	    \includegraphics[scale=0.5]{img/commits/commits iniciales.png}
	    \caption{Commits de archivos html y css iniciales}
	\end{figure}
	      
	\item Agregando secciones:
	\begin{figure}[H]
	    \centering
	    \includegraphics[scale=0.5]{img/commits/agregando secciones.png}
	    \caption{Commits agregando secciones amigos, universidad y hobbies}
	\end{figure}
        
\end{itemize}

%%%%%%%%%%%%%%%%%%%%%%%%%%%%%

\section{\textcolor{red}{Rúbricas}}
\subsection{\textcolor{red}{Entregable Informe}}
\begin{table}[H]
	\caption{Tipo de Informe}
	\setlength{\tabcolsep}{0.5em} % for the horizontal padding
	{\renewcommand{\arraystretch}{1.5}% for the vertical padding

		\begin{tabular}{|p{3cm}|p{12cm}|}
			\hline
			\multicolumn{2}{|c|}{\textbf{\textcolor{red}{Informe}}}                                                                                                      \\
			\hline
			\textbf{\textcolor{red}{Latex}} & \textcolor{blue}{El informe está en formato PDF desde Latex,  con un formato limpio (buena presentación) y facil de leer.} \\
			\hline
		\end{tabular}
	}
\end{table}

\clearpage

\subsection{\textcolor{red}{Rúbrica para el contenido del Informe y demostración}}

\begin{itemize}
	\item El alumno debe marcar o dejar en blanco en celdas de la columna \textbf{Checklist} si cumplio con el ítem correspondiente.
	\item Si un alumno supera la fecha de entrega,  su calificación será sobre la nota mínima aprobada, siempre y cuando cumpla con todos lo items.
	\item El alumno debe autocalificarse en la columna \textbf{Estudiante} de acuerdo a la siguiente tabla:

	      \begin{table}[ht]
		      \caption{Niveles de desempeño}
		      \begin{center}
			      \begin{tabular}{ccccc}
				      \hline
				                      & \multicolumn{4}{c}{Nivel}                                                              \\
				      \cline{1-5}
				      \textbf{Puntos} & Insatisfactorio 25\%      & En Proceso 50\% & Satisfactorio 75\% & Sobresaliente 100\% \\
				      \textbf{2.0}    & 0.5                       & 1.0             & 1.5                & 2.0                 \\
				      \textbf{4.0}    & 1.0                       & 2.0             & 3.0                & 4.0                 \\
				      \hline
			      \end{tabular}
		      \end{center}
	      \end{table}

\end{itemize}

\begin{table}[H]
	\caption{Rúbrica para contenido del Informe y demostración}
	\setlength{\tabcolsep}{0.5em} % for the horizontal padding
	{\renewcommand{\arraystretch}{1.5}% for the vertical padding
		%\begin{center}
		\begin{tabular}{|p{2.7cm}|p{7cm}|x{1.3cm}|p{1.2cm}|p{1.5cm}|p{1.1cm}|}
			\hline
			\multicolumn{2}{|c|}{Contenido y demostración} & Puntos                                                                                                                                                                                                          & Checklist & Estudiante & Profesor   \\
			\hline
			\textbf{1. GitHub}                             & Hay enlace URL activo del directorio para el  laboratorio hacia su repositorio GitHub con código fuente terminado y fácil de revisar.                                                                           & 2         & X          & 2        & \\
			\hline
			\textbf{2. Commits}                            & Hay capturas de pantalla de los commits más importantes con sus explicaciones detalladas. (El profesor puede preguntar para refrendar calificación).                                                            & 4         & X          & 3         & \\
			\hline
			\textbf{3. Código fuente}                      & Hay porciones de código fuente importantes con numeración y explicaciones detalladas de sus funciones.                                                                                                          & 2         & X          & 2        & \\
			\hline
			\textbf{4. Ejecución}                          & Se incluyen ejecuciones/pruebas del código fuente  explicadas gradualmente.                                                                                                                                     & 2         & X          & 2        & \\
			\hline
			\textbf{5. Pregunta}                           & Se responde con completitud a la pregunta formulada en la tarea.  (El profesor puede preguntar para refrendar calificación).                                                                                    & 2         & X          & 2        & \\
			\hline
			\textbf{6. Fechas}                             & Las fechas de modificación del código fuente estan dentro de los plazos de fecha de entrega establecidos.                                                                                                       & 2         & X          & 2        & \\
			\hline
			\textbf{7. Ortografía}                         & El documento no muestra errores ortográficos.                                                                                                                                                                   & 2         & X          & 1        & \\
			\hline
			\textbf{8. Madurez}                            & El Informe muestra de manera general una evolución de la madurez del código fuente,  explicaciones puntuales pero precisas y un acabado impecable.   (El profesor puede preguntar para refrendar calificación). & 4         & X           & 3        & \\
			\hline
			\multicolumn{2}{|c|}{\textbf{Total}}           & 20                                                                                                                                                                                                              &           & 17         &          \\
			\hline
		\end{tabular}
		%\end{center}
		%\label{tab:multicol}
	}
\end{table}

\clearpage

\section{Referencias}
\begin{itemize}
	\item \url{https://www.w3schools.com/java/default.asp}
	\item \url{https://www.geeksforgeeks.org/insertion-sort/}
\end{itemize}


\end{document}
